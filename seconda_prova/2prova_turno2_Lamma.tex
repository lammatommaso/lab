\documentclass{amsart}
\usepackage[utf8]{inputenc}
\usepackage{graphicx}
\usepackage{enumerate}
\usepackage{comment}
\usepackage{multirow}
\usepackage[a4paper]{geometry}
\geometry{ a4paper, left = 1cm, right = 1cm, top = 1cm, bottom = 1cm}

\begin{document}
    \hrule
    \begin{center}
        \textbf{ Lamma    Tommaso    0000881007        Turno    II }\\
    \end{center}
    \hrule
    \begin{center}
        {\huge Misura della caratteristica di uscita di un BJT P-N-P in configurazione a Emettitore comune}
    \end{center}
    Il circuito utilizzato per la prova è il seguente :
    \begin{center}
        \includegraphics[width = 5cm, height = 7cm]{circuito.png}.
    \end{center}
    Gli strumenti utilizzati nella prova sono:
    \begin{enumerate}[(i)]
        \item Potenziometro da 1$k\Omega$
        \item Potenziometro da 100$k\Omega$
        \item Transistor BJT 2N3906(BU) (Si PNP) 
        \item Breadboard generica
        \item Oscilloscopio GOS-652 GW
        \item Multimetro digitale FLUKE 77
        \item Generatore di tensione continua IPS 3303 ISO-TECH
    \end{enumerate}
    \newpage
    I dati misurati con corrente di base 100mA e 200mA sono:\\
    \hfill \\
    \begin{center}
        \begin{tabular}{|p{1cm}|p{1cm}|p{1cm}|p{1cm}|p{2cm}|}
            \hline
            \multicolumn{5}{|c|}{Corrente di base 100mA}\\
            \hline
            \textbf V[V] & $\delta$V[V] & I[mA] & $\delta$I[mA] & fondoscala$[V]$ \\
            \hline
            4 & 0.2 & 37.6 & 0.6 & 1\\
3.8 & 0.2 & 39.2 & 0.6 & 1\\
3.6 & 0.1 & 39.1 & 0.6 & 1\\
3.4 & 0.1 & 38.8 & 0.6 & 1\\
3.2 & 0.1 & 38.4 & 0.6 & 1\\
3 & 0.1 & 37.9 & 0.6 & 1\\
2.8 & 0.1 & 37.4 & 0.6 & 1\\
2.6 & 0.1 & 37 & 0.6 & 1\\
2.4 & 0.1 & 36.5 & 0.5 & 1\\
2.2 & 0.1 & 36.1 & 0.5 & 1\\
2 & 0.1 & 35.6 & 0.5 & 1\\
1.8 & 0.1 & 35.1 & 0.5 & 1\\
1.6 & 0.1 & 34.5 & 0.5 & 1\\
1.4 & 0.1 & 33.8 & 0.5 & 1\\
1 & 0.04 & 32.2 & 0.5 & 0.2\\
0.9 & 0.03 & 31.7 & 0.5 & 0.2\\
0.8 & 0.03 & 31.2 & 0.5 & 0.2\\
0.2 & 0.01 & 18.9 & 0.3 & 0.1\\
0.15 & 0.01 & 14.2 & 0.2 & 0.1\\
0.12 & 0.01 & 9.9 & 0.1 & 0.1\\
0.1 & 0.004 & 6.5 & 0.1 & 0.02\\
0.08 & 0.003 & 3.9 & 0.06 & 0.02\\
0.06 & 0.003 & 2.1 & 0.03 & 0.02\\
0.05 & 0.003 & 1.3 & 0.02 & 0.02\\

            \hline      
        \end{tabular}
        \hspace{1cm}
        \begin{tabular}{|p{1cm}|p{1cm}|p{1cm}|p{1cm}|p{2cm}|}
            \hline
            \multicolumn{5}{|c|}{Corrente di base 200mA}\\
            \hline
            \textbf V[V] & $\delta$V[V] & I[mA] & $\delta$I[mA]  & fondoscala$[V]$ \\
            \hline
            4 & 0.2 & 37.6 & 0.6 & 1\\
3.8 & 0.2 & 39.2 & 0.6 & 1\\
3.6 & 0.1 & 39.1 & 0.6 & 1\\
3.4 & 0.1 & 38.8 & 0.6 & 1\\
3.2 & 0.1 & 38.4 & 0.6 & 1\\
3 & 0.1 & 37.9 & 0.6 & 1\\
2.8 & 0.1 & 37.4 & 0.6 & 1\\
2.6 & 0.1 & 37 & 0.6 & 1\\
2.4 & 0.1 & 36.5 & 0.5 & 1\\
2.2 & 0.1 & 36.1 & 0.5 & 1\\
2 & 0.1 & 35.6 & 0.5 & 1\\
1.8 & 0.1 & 35.1 & 0.5 & 1\\
1.6 & 0.1 & 34.5 & 0.5 & 1\\
1.4 & 0.1 & 33.8 & 0.5 & 1\\
1 & 0.04 & 32.2 & 0.5 & 0.2\\
0.9 & 0.03 & 31.7 & 0.5 & 0.2\\
0.8 & 0.03 & 31.2 & 0.5 & 0.2\\
0.2 & 0.01 & 18.9 & 0.3 & 0.1\\
0.15 & 0.01 & 14.2 & 0.2 & 0.1\\
0.12 & 0.01 & 9.9 & 0.1 & 0.1\\
0.1 & 0.004 & 6.5 & 0.1 & 0.02\\
0.08 & 0.003 & 3.9 & 0.06 & 0.02\\
0.06 & 0.003 & 2.1 & 0.03 & 0.02\\
0.05 & 0.003 & 1.3 & 0.02 & 0.02\\

            \hline
        \end{tabular}
    \end{center}
    I loro rispettivi grafici sono:\\
    \hfill \\
    \begin{center}
        \includegraphics[width = 9cm, height = 9cm]{ib100/grafico_transistor.png}
        \hspace{0.1cm}
        \includegraphics[width = 9cm, height = 9cm]{ib200/grafico_transistor.png}
    \end{center}
    I risultati finali sono:\\
\end{document}